\chapter{Conclusion}
\label{conclusion}

	The aim of this project was to gain an insight into the effects of non-Euclidean geometry on a users sense of immersion in virtual environments, more specifically when in Virtual Reality.
	To do this, test scenes were created using both Euclidean and non-Euclidean geometry, which would be shown to users as a way to gather feedback on various metrics surrounding their sense of immersion in the scenes. % TODO: Reword this? make it longer?

	Results from the experiments conducted indicate that immersion is possible in non-Euclidean virtual environments.
	As well as this, the data from the experiments show that a user's sense of immersion can be increased compared to a standard Euclidean environment, if they transition from the standard environment to a non-Euclidean one.

	The outcomes of the experiments show that video games which make use of VR systems could additionally benefit from the use of non-Euclidean space, as a way to both expand the features of the game and increase user immersion at the same time.

	The results from the experiments open up room for future research into the subject area, such as the effects different control methods like a gamepad or room-scale motion tracking have on a users sense of immersion in a non-Euclidean environment, or perhaps how different uses of non-Euclidean space individually effect a users sense of immersion.
