\section{Conclusion}
\label{lr:conclusion}

\begin{multicols*}{2}
	In Virtual Reality, non-standard geometry is definitely an area with potential to engage, and possibly spark an interest within a user about the capabilities and uses of such concepts.
	
	This paper has given an overview into the various components required for the creation of an environment suitable for use in VR, how non-standard geometrical worlds can be created for use in a VE, and how the areas both require adaptation in order to work together.
	
	In terms of Virtual Reality, existing literature is very thorough in its coverage of the possibilities and effects of its use both within and outside of video game environments. Although there is definitely still room for further research into the more niche areas of study, there is a solid foundation for use as a reference for development in the field.
	
	For non-Euclidean geometry, however, there is a distinct lack of academic research into its use within virtual environments, with most studies focusing instead on the theoretical applications for it. Reports around the use of non-standard geometry in video games does exist, however due to the non-scholarly nature of the sources, which are mainly video game journalist articles or developer interviews, they cannot be completely trusted for academic purposes.
	
	Due to combination of the findings surrounding these areas, future research into the use of non-Euclidean geometry within a VR system is very open to exploration, and as such, research would be beneficial as a possible basis for further study, perhaps even encouraging it.
\end{multicols*}