% Hide chapter number
\setcounter{secnumdepth}{-2}

\chapter{Appendix}
\label{appendix}

% Reset chapter numbering so referencable objects in the section are labelled appropriately
\setcounter{chapter}{2}
\renewcommand{\thechapter}{\Alph{chapter}}
\setcounter{secnumdepth}{2}

% Give the appendix sections Alpha numbering
\renewcommand\thesection{\Alph{section}}

	% List of Figures
	\begingroup
		\section{List of Figures}
		\label{appendix:figures}

		Below is a list of the figures which are present in the document, along with their corresponding page number.

		\renewcommand{\chapter}[2]{}		% Change chapter behaviour to make it not force the list of figures onto the following page
		\listoffigures						% Display the list of figures
	\endgroup

\newpage
	\section{Code Samples}
	\label{appendix:code}

		Below are code snippets for sections or complete functions that have specifically been referenced in the body of the report.
		The full source code and project can be found in the \enquote{Product} directory in the GitHub Repository found below:

		\url{https://www.github.com/nboxhallburnett/IndividualProject}

		\begin{lstlisting}[caption="Camera Positioning - CameraRenderPosition.cs", label=appendix:code:camera, firstnumber=85]
// Calculate the offset vector of the players camera from the perspective point
Vector3 offset = PointOfView.position - _player.position + ((_player.position - _player.parent.parent.position) / 2);

// Position and rotate the cameras depending on the type of illusion they are going for
if (!Inverse) {
	Moveable.position = Helper.RotatePointAroundPivot(RenderPosition.position - offset, RenderPosition.position, _relativePortalRot.eulerAngles);
	Moveable.rotation = _relativePortalRot * Quaternion.Euler(rotationOffset + _defaultRot - _normalisedDefaultRot);
} else {
	Moveable.position = RenderPosition.position - offset;
	Moveable.rotation = _relativePortalRot * Quaternion.Euler((RenderPosition.transform.up == Vector3.up ? rotationOffset : -rotationOffset) + _defaultRot - _normalisedDefaultRot + new Vector3(0, 180f, 0));
}

// Set the near clipping plane of the camera to only render starting from the closest visible area
_cam.nearClipPlane = (Helper.FindClosestPoint(_bounds, transform.position) - transform.position).magnitude / 2f;
		\end{lstlisting}

		\begin{lstlisting}[caption="Player Positioning - CameraRenderPosition.cs", label=appendix:code:player, firstnumber=125]
/// <summary>
/// Re-Position the player from their current position to the equivelant position of its point of view, depending on their relative position
/// </summary>
/// <param name="player">Player to potentially transport</param>
/// <param name="centre">Centre of the currently colliding object</param>
/// <param name="forward">Forward vector of the currently colliding object</param>
public void positionPlayer (Collider player, Vector3 centre, Vector3 forward) {
	// Get the distance vector between the player and the centre of the currently colliding object
	Vector3 distance = player.transform.position - centre;
	Quaternion inverseFlip = Quaternion.Euler(0, 0, 0);

	// If only one of the points are inverted, make sure to flip the player
	if (Inverse != _linkedScript.Inverse) {
		inverseFlip = Quaternion.Euler(0, 180f, 0);
	}

	// If the player is more than half way through the object, transport them to the linked area
	if (Vector3.Dot(distance.normalized, forward) < 0) {
		// Set player position
		distance = Helper.RotatePointAroundPivot(distance, Vector3.zero, (_relativePlayerRot.eulerAngles + inverseFlip.eulerAngles));
		player.transform.position = PointOfView.position;
		player.transform.position += distance;

		// Set player rotation
		rotationOffset += _relativePlayerRot.eulerAngles + inverseFlip.eulerAngles;
		player.transform.rotation = _relativePlayerRot * inverseFlip * player.transform.rotation;

		// Update the players momentum
		_playerControl.UpdateMoveThrottle(_relativePlayerRot * inverseFlip * _playerControl.GetMoveThrottle());
	}
}
		\end{lstlisting}

		\begin{lstlisting}[caption="Render Cull Shader - Subtractive.shader", label=appendix:code:shader]
Shader "NEGeo/Subtractive" {
	SubShader{
		// Render just before the Overlay layer, to make sure it is rendered on top of other objects
		Tags{ "Queue" = "Overlay-1" }

		// Don't draw in the RGBA channels
		ColorMask 0
		// Only write to the Z-Buffer
		ZWrite On

		// No need to do any additional processing in the pass
		Pass { }
	}
}
		\end{lstlisting}

\newpage
	\section{Questionnaire}
	\label{appendix:question}
		A blank copy of the form used to gather the feedback discussed in Chapter \ref{exp} can be seen in the following two pages. Either the 1 or 2 were highlighted under the 'Experiment' section at the top of each page before being given to a participant. This was used as a reference for myself as to which experiment the form was for, without indicating to the participant the nature of the experiment.

		A digital transcription of the gathered data can be found in the \enquote{Raw Data} sheet in the \href{https://github.com/nboxhallburnett/IndividualProject/blob/master/Report/Resources/Feedback\%20Data.xlsx?raw=true}{\texttt{Report/Resources/Feedback Data.xlsx}} file inside the Git repository mentioned in \autoref{appendix:code}.
		The completed physical versions of the forms containing the raw data gathered are available upon request.

		\includepdf[pages={-},scale=1]{"Resources/Project Feedback Questionnaire"}

\newpage
	\section{Ethics Review}
	\label{appendix:ethics}
		A completed and signed copy of the Project Ethical Review Form can be found on the following two pages.

		\includepdf[pages={-},scale=0.9]{"../Ethical Review Scan"}
