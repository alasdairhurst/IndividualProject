\section{Presence in Virtual Reality}
\label{lr:vr}

%\begin{multicols*}{2}
	
	\subsection{Introduction}
	\label{lr:vr:intro}
	%		This section will introduce the key concepts that will be discussed in the following subsections.
		Presence is the term used in VR systems to describe a user's sense of existing inside a VE. Because of this, it can be seen as one of, if not the most important feature of the design for a VR system.
		
		There are a few main features which are key to achieving a sense of presence in a VE. One example would be the general perception of the user for things such as depth perception, the ability for a person to calculate the distance of a point is from themselves, and sensorimotor adaptation, which is the calibration of a users senses to fit their environment. 
		As well as this, there are specified areas such as affordance detection, the ability for a person to calculate interaction with an object or environment, which helps assist a user with navigation and interaction in a VE. 
		The following sections will review existing literature on the aforementioned features.
	
	\subsection{Affordances in VR}
	\label{lr:vr:affordances}
	%		Here I will introduce the concept of, and cover research into affordances in Virtual Reality.
		%Perceiving affordances in virtual reality: influence of person and environmental properties in perception of standing on virtual grounds
		Affordance is especially important to get right in VR, particularly in a video game environment, due to the world a user is in is designed to be interacted with. Unlike what common sense would dictate, research on affordance in the real world may not be as applicable for when designing interactions in a virtual world as expected. A study \cite{Regia-Corte2012} focusing on a persons perception for whether or not they could stand upright on an object at various angles in a virtual world found that, contrary to expected results, users would tend to be more cautious when judging their ability in a virtual environment (critical angle 21.98\degree) compared to an equivalent object in the real world (critical angle 30\degree), even with a lack of apparent risk. 
		This study covered the various requirements for the experiments well, however there were a few areas in which it could still be expanded upon. These areas include comparing results from using a variety of VR systems to judge responses based off on hardware response times, display resolutions, and refresh rates, as well as increases and decreases in model and texture quality (not just the texture itself like they did) for judging the surfaces of the objects.
		
		% Designing presence for real locomotion in immersive virtual environments: an affordance-based experiential approach
		Another way in which affordance directly affects a users sense of presence in VR is the simulation of real locomotion for the user. In a study \cite{Turchet2015} covering the ways in which a users movement is recorded and displayed in a VE (including, but not limited to, individual foot tracking, arm, leg, and head tracking, movement speed tracking, etc.), results found that while an increase in the coverage of the input also increased the sense of presence for the user, it was more important to focus the areas of which to track on their relevance to the scenario at hand. As well as this, the study found that while the input tracking itself was important, to attain a greater sense of presence it was also required for the physical representation of the user to match their real physical selves as much as possible, such as gender, height, and weight.
		This study, while extremely thorough in its coverage of the various possible elements required to generate a sense of presence, could have recorded and displayed its results in a more usable manner than it was. Due to the way the experiments were conducted, the results do not indicate any sense of importance for each individual step in its contribution to the feeling of presence in the user. Because of this, further research could be undertaken to create a weighted system designed around the experiments conducted, signifying its importance for specific VR systems, creating a more convenient set of data for future use.
		
	\subsection{Perception in Virtual Environments}
	\label{lr:vr:perception}
		This section will cover existing research into how people perceive environments in Virtual Reality.
		% Using virtual reality to augment perception, enhance sensorimotor adaptation, and change our minds.
		Perception is a key feature to get right in VR, as incorrect calibration for features such as head tracking and input response can have both unintended, and undesired side-effects. In studies \cite{Wright2006}  \cite{Wright2009} \cite{Wright2011} \cite{Wright2013} \cite{Wright2014} around the effects of perception augmentation and sensorimotor adaptations, results showed that tests around sensorimotor processes that made use of Virtual Reality systems could have both intended and unintended effects on the participants central nervous system. 
		(NOTE) This paragraph needs expanding upon
		
		
		% Immersive Virtual Environment Technology to Supplement Environmental Perception, Preference and Behavior Research: A Review with Applications
		
		(PLACEHOLDER) Discuss \cite{smith2015} - Immersive virtual environment technology to supplement environmental perception, preference, and behaviour research
		
	\subsection{Conclusion}
	\label{lr:vr:conclusion}
		Here I will summarise the concepts and sources discussed in the previous sections, how reliable they may be, and how the concepts could apply to my project.
		Due to recent relevant technology being both of a high quality, and low enough cost to warrant consumer interest, as well as its applications in not just Video Games, but in Medicine, Psychology, and Education, Virtual Reality is a very highly researched area.
		
		Studies have been conducted to cover an extremely wide range of applications and effects for VR, and because of this, there is a very solid foundation of work which can be both applied for practical uses, as well as a platform to build upon for future research.
		
		This is especially true in the two main focus points for the literature covered in this review, and will be beneficial for reference during the creation of the proposed system.
	
%\end{multicols*}
