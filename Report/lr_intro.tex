\section{Introduction}
\label{lr:intro}
%	Here I will be introducing and outlining the concepts that will be covered in the following sections.
	
\begin{multicols*}{2}
	Virtual Reality (VR) is fast becoming a mainstream platform for video games, with high quality devices allowing the feeling of \enquote{presence} in a game world being released for computers, game consoles, and mobile devices alike, creating a vast user base for potential products.

	As well as this, the popularity of video games which utilise non-standard or real-world geometric principles are also increasing, with the popularity of titles such as Antichamber \cite{Antichamber2013} and Portal 2 \cite{Portal22011} engaging consumers with their non-conformity to the physics and geometry of average game worlds.

	This review explores the existing literature that covers the various requirements for creating a VE that consumers would perceive as \enquote{realistic} in VR.
	As well as this, it also covers how these requirements have been, and can be, adapted to work inside of a world which does not follow the restrictions of real world geometry, such as in non-Euclidean or Escheresque space.
	Gaps in the current literature are outlined, alongside prospective areas for future research.

	This review is split up into three chapters, each one covering a specific area of existing literature relevant to it:
	\begin{enumerate}
		\item \nameref{lr:vr}, which will cover various elements creating a sense of presence in a Virtual Environment (VE) using VR requires, such as user perception and affordances
		\item \nameref{lr:ne}, which will cover existing applications of non-standard geometry in video game systems
		\item \nameref{lr:cross}, which will cover how the requirements from the previous chapters are to be utilised together to form a working system, as well as the tools which are best suited for constructing such a system
	\end{enumerate}

\end{multicols*}
