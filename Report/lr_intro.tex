\section{Introduction}
\label{lr:intro}

	This review explores the existing literature that covers the various requirements for creating a VE that consumers would perceive as \enquote{realistic} in VR.
	As well as this, it also covers how these requirements have been, and can be, adapted to work inside of a world which does not follow the restrictions of real world geometry, such as in non-Euclidean or Escheresque space.
	Gaps in the current literature are outlined, alongside prospective areas for future research.

	There are two main areas which make up the requirements for the aims of this study.
	The first area is identifying the components which are required to achieve a sense of presence in a Virtual Environment (VE) when in VR, and how they influence a users feeling of immersion within it.
	The second area covers implementations of non-Euclidean geometry in existing video game systems, as well as the tools which were used to develop them. % I did start adding something here, if you remember what it was then have at it

	The following sections in this chapter cover both the aforementioned areas, as well as a conclusion which summarises the reviewed literature and its impacts on the proposed project.
